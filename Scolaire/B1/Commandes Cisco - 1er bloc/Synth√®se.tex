\documentclass[a4paper]{article}

\usepackage{lmodern}

%% Language and font encodings
\usepackage[french]{babel}
\usepackage[utf8x]{inputenc}
\usepackage[T1]{fontenc}

%% Sets page size and margins
\usepackage[a4paper,top=3cm,bottom=3cm,left=2cm,right=2cm,marginparwidth=2cm]{geometry}
%% Useful packages
\usepackage{amsmath}
\usepackage{graphicx}
\usepackage[colorinlistoftodos]{todonotes}
\usepackage[colorlinks=true, allcolors=black]{hyperref}
\usepackage{fourier-orns}
\usepackage{titlesec}
\usepackage{fancyhdr}
\usepackage{fancyvrb}
%\renewcommand{\thefootnote}{\*}
\pagestyle{fancy} 
\setcounter{tocdepth}{5}



%% Tikz stuff
\usepackage{tikz}
\usetikzlibrary{calc, arrows}
\tikzstyle{incolore} = [rectangle, rounded corners, draw=black, minimum height=1cm, minimum width=3cm, text width=3cm, text centered]


\usepackage{float}
\usepackage{libertine}
\newcommand{\hsp}{\hspace{20pt}}
\newcommand{\HRule}{\rule{\linewidth}{0.5mm}}




% Défini la règle d'en-tête
\renewcommand{\headrulewidth}{1pt}
\fancyhead[C]{} 
\fancyhead[L]{}
\fancyhead[R]{\footnotesize{\leftmark}}

% Défini la règle de fond de page
\renewcommand{\footrulewidth}{1pt}
\fancyfoot[C]{} 
\fancyhead[L]{}
\fancyfoot[R]{\thepage}

\definecolor{Zgris}{rgb}{0.87,0.85,0.85}

\usepackage{eso-pic,graphicx}
\usepackage{xcolor}
\newcommand{\bgimg}[1]{
\AddToShipoutPicture
   {
      \put(\LenToUnit{0 cm},\LenToUnit{0 cm})
      {
            \includegraphics[width=\paperwidth,height=\paperheight]{#1} 
      }
   }
}

\begin{document}
  \begin{titlepage}
    \begin{sffamily}
      \begin{center}
        \textnormal{}\\[6.5cm]
        % Upper part of the page. The '~' is needed because \\
        % only works if a paragraph has started.
        % Title
        \HRule \\[0.4cm]
        { \Huge \bfseries Synthèse\\ Commandes Cisco\\ [0.4cm] }
        \HRule \\[3cm]
        \Large
        Premier Bloc\\
        Sécurité des systèmes\\
        Année académique 2019-2020\\[0.5cm]
        \emph{Rédigé par Sénéchal Julien}
        \vfill
        % Bottom of the page
        {\large 22 Mars 2020}
      \end{center}
    \end{sffamily}
  \end{titlepage}

\Large\textbf{ATTENTION - LORSQUE VOUS VOYEZ "[]", CELA SIGNIFIE QUE VOUS DEVEZ MODIFIER CETTE VALEUR EN FONCTION DE CE QUI Y EST NOTÉ}
\normalsize
\section{Les Routeurs}

\subsection{Commandes de bases}
\begin{itemize}
  \item Se mettre en mode \emph{EXEC} : \emph{enable}
  \item Configurer le terminal : \emph{configure terminal}
  \item Changer le nom de notre routeur : \emph{hostname [nom]}
  \item Sortir de n'importe quel mode/environnement (ex: EXEC, Configure terminal, interfaces,...) : \emph{exit}
  \item Vérifier tous les ports du routeur : \emph{show ip interface brief}
  \item Pour connaître les protocoles utilisés (ex : \emph{rip}): \emph{show ip protocols}
\end{itemize}

\subsection{Effacer la configuration d'un routeur}
Vous devez être en mode \emph{EXEC} pour cela :
\begin{itemize}
  \item \emph{erase startup-config}
  \item \emph{reload}
\end{itemize}

\subsection{Commandes interfaces}
\begin{itemize}
  \item Configurer une interface : \emph{interface [nom\_interface]}
  \item Changer l'adresse IP : \emph{ip address [IP] [NetMask]}
  \item Décrire une interface : \emph{description [La description]}
  \item Garder l'interface active (obligatoire pour utiliser une interface): \emph{no shutdown}
  \item Vitesse de la clock (uniquement sur le DCE): \emph{clock rate [vitesse]}
\end{itemize}

\subsection{Créer des routes}
\begin{itemize}
  \item Route récursive (On choisit la cible en fonction de son IP) :\\ 
  \emph{ip route [IP de destination des paquets] [NetMask du réseaux destinataire] [IP routeur]}
  \item Route directement connectée (On envoie les paquets directement dans un \emph{Serial}) :\\ 
  \emph{ip route [IP de destination des paquets] [NetMask du réseaux destinataire] [serial connecté au routeur]}
\end{itemize}

\subsection{Logging Synchronous}
On peut recevoir des messages du routeur qui trouble l’entrée des commandes.
Pour éviter ça, on peut demander un \emph{Logging Synchronous}.\\
Pour ce faire, une fois dans \emph{Router(config\#)} :
\begin{itemize}
  \item \emph{Line console 0}
  \item \emph{logging synchronous}
  \item \emph{exit}
\end{itemize}

\subsection{Sécurité}
\begin{itemize}
  \item Pour mettre un mode de passe au \emph{EXEC MODE} : \emph{enable secret [mdp]}
  \item Pour encrypter les mots de passe dans le fichier de configuration : \emph{service password-encryption}
  \item Pour choisir le nombre de caractère minimum pour tous les mots de passe du routeur : \\\emph{security passwords min-length [nb de caractère]}
  \item Pour empêcher les attaques par force brute, nous pouvons mettre un cooldown entre chaque erreur : \emph{login block-for [secondes] attempts [nombre de tentatives] within [secondes]}.\\
  Le premier champ sert a indiquer le nombre de seconde de cooldown entre chaque tentative. Le dernier champ sert à indiquer le nombre de secondes que l'utilisateur peut prendre pour renseigner son mot de passe.
  \item Pour pouvoir permettre de se déconnecter du routeur automatiquement après un temps d'inactivitée :\\
  \emph{line console 0}\\
  \emph{exec-timeout [minutes] [secondes]}\\
  \emph{live vty 0 4}\\
  \emph{exec-timeout [minutes] [secondes]}\\
  \emph{exit}
\end{itemize}

\subsection{Etablir une connection SSH}
Il faut assigner le nom de domaine : \\
\emph{ip domain-name [nom de domaine]}\\
Puis, il faut créer un utilisateur pour le SSH. Si le privilège de l'utilisateur n'est pas spécifié dans la commande,
celui-ci aura automatiquement les privilèges \emph{EXEC} (niveau 15).\\
\emph{username [nom] privilege [level] secret [MotDePasse]}\\
Maintenant, nous allons configurer notre "VTY line" afin d'accepter le SSH.\\
\emph{line vty 0 4}\\
Ici le "0 4" va permettre l'ouverture de 5 sessions maximum. (Ne me demandez pas pourquoi, c'est juste le fruit de mes recherches.)
Maintenant il faut renseigner par quel moyen on veut se connecter\\
\emph{transport input ssh}\\
\emph{login local}\\
\emph{exit}\\
\emph{crypto key generate rsa}\\
Après cela, on nous demande de rentrer le nombre de bits pour le modulus, entrez \emph{"1024"}\\

\subsection{Configuration RIP}
Rentrez en mode EXEC, puis en mode configuration du terminal.\\
Entrez \emph{routeur rip} pour commencer la configuration du protocole RIP.\\
Faites bien attention a marquer \emph{version 2} par après, sans quoi votre protocole ne prendra pas en 
compte les NetMasks !\\
\begin{itemize}
  \item Définir les interfaces concernées par le RIP : \emph{network [IP]}
  \item Définir les interfaces n'étant pas connectée à un routeur (par Sécurité) : \emph{passive-interface [interface (ex : g0/0)]}
  \item Désactiver le résumé de routes : \emph{no auto-summary}
\end{itemize}
\textcolor{red}{\textbf{ATTENTION :}} Le réseaux "Internet" ne participe pas au routage RIP !
Pour \emph{Internet}, nous utiliserons une route statique !\\
Ensuite, afin de pouvoir propager notre route vers \emph{Internet} sur tout le RIP, nous allons devoir configurer cela :
\begin{itemize}
  \item \emph{routeur rip} (pour ce remettre en configuration RIP)
  \item \emph{default-information originate} (Seulement sur le routeur qui nous a servi a configurer notre route !)
\end{itemize}
Pour vérifier que tout est opérationnel : \emph{show ip route} ou \emph{show ip route rip}\\
Au cas où vous constatez un problèmes avec vos routes, et voulez recommencer, la commande \emph{clear ip route *} va supprimer 
toutes les routes présentes sur \textbf{le} routeur sur lequel vous avez lancer la commande !

\subsection{IPV6}
\begin{itemize}
  \item Pour activer l'IPV6 sur le routeur : \emph{ipv6 unicast-routing}
  \item Définir l'adresse : \emph{ipv6 address [adresse + CIDR] eui-64}\\
  (le protocole \emph{eui-64} va permettre en quelque sorte de remplacer 
  un DHCP en permettant automatiquement de transmettre aux machines sur le réseau l'adresse du réseau et va ainsi permettre aux machines de créer leurs
  adresses IPV6 sur base de leur adresse MAC pour qu'il n'y aie pas de répétition. L'adresse obtenue sur les PC se nomme le \emph{SLAAC}. Donc, utile uniquement sur les interfaces LAN.)
  \item On n'oublie pas le \emph{no shutdown} 
\end{itemize}
\subsubsection{Routage statique en IPV6}
\begin{itemize}  
  \item Routes directement connectée : \emph{ipv6 route [adresse + CIDR] [interface]}
  \item Routes récursives : \emph{ipv6 route [adresse + CIDR] [adresse routeur]}
  \item Routes par défaut : \emph{ipv6 route $::$/0 [interface]}\\
  Le \emph{"$::$/0"} signifie la même chose que le 0.0.0.0 0.0.0.0 en IPV4
  \item Pour supprimer une route :\\ \emph{no ipv6 route [adresse + CIDR] [adresse routeur/interface (selon si récursive ou directement connectée)]}
\end{itemize}
\subsubsection{RIPng}
\begin{itemize}
  \item Pour activer le RIPng sur votre routeur, faites \emph{ipv6 rip [nom-de-domaine] enable} dans la configuration de vos interfaces. Le "nom de domaine" peut être ce que vous voulez si non précisé dans les consignes du moment du vous utilisiez
  le même sur vos autres routeurs faisant partie de votre RIPng. Ne pas mettre cette commande sur les interfaces LAN pour des raisons de sécurité.
  \item Une fois le RIPng activé sur toutes les interfaces non-LAN de vos routeurs, afin que les interfaces LAN puissent s'en servir, entrez la commande \emph{ipv6 router rip [nom-de-domaine]} dans la configuration du terminal,
  puis faites \emph{redistribute connected}.
  \item De la même manière que sur le RIPv2, la commande \emph{ipv6 rip [nom-de-domaine] default-information originate} vous permettra de définir une route par défaut envoyant par exemple sur \emph{internet}. 
  Uniquement à mettre sur l'interface du routeur vers lequel doivent sortir les paquets de cette route par défaut.
  \item Pour voir les routes créés a partir de votre rip et vos routes statiques, faites \emph{show ipv6 route}
\end{itemize}








\newpage
\section{Les Switchs}

\subsection{Les commandes de bases}
\begin{itemize}
  \item Se mettre en mode \emph{EXEC} : \emph{enable}
  \item Sortir de n'importe quel mode/environnement (ex: EXEC, Configure terminal, interfaces,...) : \emph{exit}
  \item Vérifier tous les ports du switch : \emph{show ip interface brief}
  \item Pour éteindre plusieurs interfaces en même temps : \\
  \emph{interface range [range1] , [range2] , [range3]}\\
  \emph{shutdown}\\
  Exemple :\emph{ interface range FastEthernet0/1–4 , GigabitEthernet0/1-2}
  
\end{itemize}

\subsection{Supprimer la configuration d'un Switch}
Vous devez être en mode \emph{EXEC} pour cela :
\begin{itemize}
  \item \emph{delete {\NoAutoSpacing flash:vlan.dat}}
  \item \emph{erase startup-config}
  \item \emph{reload}
\end{itemize}

\subsection{Etablir une connection SSH}
Il faut assigner le nom de domaine : \\
\emph{ip domain-name [nom de domaine]}\\
Puis, il faut créer un utilisateur pour le SSH. Si le privilège de l'utilisateur n'est pas spécifié dans la commande,
celui-ci aura automatiquement les privilèges \emph{EXEC} (niveau 15).\\
\emph{username [nom] privilege [level] secret [MotDePasse]}\\
Maintenant, nous allons configurer notre "VTY line" afin d'accepter le SSH.\\
\emph{line vty 0 15}\\
Ici le "0 15" va permettre l'ouverture de 16 sessions maximum.
Maintenant il faut renseigner par quel moyen on veut se connecter\\
\emph{transport input ssh}\\
\emph{login local}\\
\emph{exit}\\
\emph{crypto key generate rsa}\\
Après cela, on nous demande de rentrer le nombre de bits pour le modulus, entrez \emph{"1024"}\\

\subsection{Sécurité}
\begin{itemize}
  \item Pour pouvoir permettre de se déconnecter du routeur automatiquement après un temps d'inactivitée :\\
  \emph{line console 0}\\
  \emph{exec-timeout [minutes] [secondes]}\\
  \emph{live vty 0 15}\\
  \emph{exec-timeout [minutes] [secondes]}\\
  \emph{exit}
  \item Pour empêcher les attaques par force brute, nous pouvons mettre un cooldown entre chaque erreur : \emph{login block-for [secondes] attempts [nombre de tentatives] within [secondes]}.\\
  Le premier champ sert a indiquer le nombre de seconde de cooldown entre chaque tentative. Le dernier champ sert à indiquer le nombre de secondes que l'utilisateur peut prendre pour renseigner son mot de passe.
\end{itemize}



















\end{document}