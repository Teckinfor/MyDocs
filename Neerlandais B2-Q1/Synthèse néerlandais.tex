\documentclass[a4paper]{article}

\usepackage{lmodern}

%% Language and font encodings
\usepackage[french]{babel}
\usepackage[utf8x]{inputenc}
\usepackage[T1]{fontenc}
\usepackage{enumitem}
\usepackage{xcolor}
\usepackage{pifont}

%% Sets page size and margins
\usepackage[a4paper,top=3cm,bottom=3cm,left=2cm,right=2cm,marginparwidth=2cm]{geometry}
%% Useful packages
\usepackage{amsmath}
\usepackage{graphicx}
\usepackage[colorinlistoftodos]{todonotes}
\usepackage[colorlinks=true, allcolors=black]{hyperref}
\usepackage{fourier-orns}
\usepackage{titlesec}
\usepackage{fancyhdr}
\usepackage{fancyvrb}
%\renewcommand{\thefootnote}{\*}
\pagestyle{fancy} 
\setcounter{tocdepth}{5}
\usepackage{array}



%% Tikz stuff
\usepackage{tikz}
\usetikzlibrary{calc, arrows}
\tikzstyle{incolore} = [rectangle, rounded corners, draw=black, minimum height=1cm, minimum width=3cm, text width=3cm, text centered]
\usepackage{float}

\usepackage{makecell}
\usepackage{libertine}
\newcommand{\hsp}{\hspace{20pt}}
\newcommand{\HRule}{\rule{\linewidth}{0.5mm}}





\renewcommand{\headrulewidth}{1pt}
\fancyhead[C]{} 
\fancyhead[L]{}
\fancyhead[R]{\footnotesize{\leftmark}}

\renewcommand{\footrulewidth}{1pt}
\fancyfoot[C]{} 
\fancyhead[L]{}
\fancyfoot[R]{\thepage}

\definecolor{Zgris}{rgb}{0.87,0.85,0.85}

\usepackage{eso-pic,graphicx}
\usepackage{xcolor}
\newcommand{\bgimg}[1]{
\AddToShipoutPicture
   {
      \put(\LenToUnit{0 cm},\LenToUnit{0 cm})
      {
            \includegraphics[width=\paperwidth,height=\paperheight]{#1} 
      }
   }
}


\begin{document}


%%\bgimg{Image_15.jpg}





  \begin{titlepage}
    \begin{sffamily}
    \begin{center}
      \textnormal{}\\[6.5cm]
      % Upper part of the page. The '~' is needed because \\
      % only works if a paragraph has started.
      % Title
      \HRule \\[0.4cm]
      { \Huge \bfseries Synthèse\\ Neerlandais B2-Q1\\ [0.4cm] }
      \HRule \\[3cm]
      \Large
      Deuxième Bloc\\
      Sécurité des systèmes\\
      Année académique 2020-2021\\[0.2cm]
      \emph{Rédigé par Sénéchal Julien}
      \vfill
      % Bottom of the page
      {\large 24 Décembre 2020}
    \end{center}
    \end{sffamily}
  \end{titlepage}

\section{Temps primitif}
\begin{itemize}[label=\textbullet, font=\Large]
  \item Tout temps primitif se termine par \emph{EN} au participe passé
  \begin{center}
    \begin{tabular}{|c|c|c|c|}
      \hline
      \textbf{Français} & \textbf{Verbe} & \textbf{Imparfait} & \textbf{Participe passé}\\
      \hline
      être assis & zitten & zat/zaten & gezeten\\
      \hline
      mettre & zetten & zette & gezet\\
      \hline
      s'allonger & ligten & lag/lagen & gelegen\\
      \hline
      poser & legden & legde & gelegd\\
      \hline
      voir & zien & zag/zagen & gezien\\
      \hline
      dire & zeggen & zei/zeiden & gezegd\\
      \hline
    \end{tabular}\\
  \end{center}
  \item IJ - EE - ExEE
  \begin{center}
    \begin{tabular}{|c|c|c|c|}
      \hline
      \textbf{Français} & \textbf{Verbe} & \textbf{Imparfait} & \textbf{Participe passé}\\
      \hline
      comprendre & begrijpen & begreep/begrepen & begrepen\\
      \hline
      mordre & bijten & beet/beten & gebeten\\
      \hline
      paraître & (b)lijken & (b)leek/(b)leken & ge(b)leken\\
      \hline
      rester & blijven & bleef/bleven & gebleven\\
      \hline
      regarder & kijken & keek/keken & gekeken\\
      \hline
      souffrir de & lijen aan & leed/leden & geleden\\
      \hline
      Rouler & rijden & reed/reden & gereden\\
      \hline
      Sembler/Briller & Schijnen & scheen/schenen & geschenen\\
      \hline
      ecrire & schrijven & schreef/schreven & geschreven\\
      \hline
      couper & snijden & sneed/sneden & gesneden\\
      \hline
      augmenter & stijgen & steeg/stegen & gestegen\\
      \hline
      etc... & etc... & etc... & etc...\\
      \hline
    \end{tabular}\\
  \end{center}
  \item IE - OO - OxEN
  \begin{center}
    \begin{tabular}{|c|c|c|c|}
      \hline
      \textbf{Français} & \textbf{Verbe} & \textbf{Imparfait} & \textbf{Participe passé}\\
      \hline
      offrir & bieden & bood/boden & geboden\\
      \hline
      profiter & genieten & genoot/genoten & genoten\\
      \hline
      mentir & liegen & loog/logen & gelogen\\
      \hline
      interdire & verbieden & verbood/verboden & verboden\\
      \hline
      perdre & verliezen & verloor/verloren & verloren\\
      \hline
      voler & vliegen & vloog/vlogen & gevlogen\\
      \hline
    \end{tabular}\\
  \end{center}
  \item UI - OO - OxEN
  \begin{center}
    \begin{tabular}{|c|c|c|c|}
      \hline
      \textbf{Français} & \textbf{Verbe} & \textbf{Imparfait} & \textbf{Participe passé}\\
      \hline
      décider & besluiten & besloot/besloten & besloten\\
      \hline
      sentir & ruiken & rook/roken & geroken\\
      \hline
      fermer & sluiten & sloot/sloten & gesloten\\
      \hline
    \end{tabular}\\
  \end{center}
  \item I - O - OxEN
  \begin{center}
    \begin{tabular}{|c|c|c|c|}
      \hline
      \textbf{Français} & \textbf{Verbe} & \textbf{Imparfait} & \textbf{Participe passé}\\
      \hline
      commencer & beginnen & begon/begonnen & begonnen\\
      \hline
      lier & binden & bond/bonden & gebonden\\
      \hline
      boire & drinken & dronk/dronken & gedronken\\
      \hline
      trouver & vinden & vond & gevonden\\
      \hline
      gagner & winnen & won/wonnen & gewonnen\\
      \hline
      chanter & zingen & zong/zongen & gezongen\\
      \hline
    \end{tabular}\\
  \end{center}
  \item E - O - OxEN
  \begin{center}
    \begin{tabular}{|c|c|c|c|}
      \hline
      \textbf{Français} & \textbf{Verbe} & \textbf{Imparfait} & \textbf{Participe passé}\\
      \hline
      toucher(atteindre) & treffen & trof/troffen & getroffen\\
      \hline
      tirer & trekken & trok/trokken & getrokken\\
      \hline
      partir & vertrekken & vetrok/vertrokken & vertrokken\\
      \hline
      envoyer & zenden & zond/zonden & gezonden\\
      \hline
      nager & zwemmen & zwom/zwommen & gezwommen\\
      \hline
    \end{tabular}\\
  \end{center}
  \item E - IE - O
  \begin{center}
    \begin{tabular}{|c|c|c|c|}
      \hline
      \textbf{Français} & \textbf{Verbe} & \textbf{Imparfait} & \textbf{Participe passé}\\
      \hline
      aider & helpen & hielp/hielpen0 & geholpen\\
      \hline
      mourir & sterven & stierf/stierven & gestorven\\
      \hline
      jeter & werpen & wierp/wierpen & geworpen\\
      \hline
    \end{tabular}\\
  \end{center}
  \item EE - A/AA - OxEN
  \begin{center}
    \begin{tabular}{|c|c|c|c|}
      \hline
      \textbf{Français} & \textbf{Verbe} & \textbf{Imparfait} & \textbf{Participe passé}\\
      \hline
      casser & breken & brak/braken & gebroken\\
      \hline
      voler(dérober) & stelen & stal/stalen & gestolen\\
      \hline
      parler & spreken & sprak/spraken & gesproken\\
      \hline
      prendre & nemen & nam/namen & genamen\\
      \hline
      peser (exception) & wegen & woog/wogen & gewogen\\
      \hline
    \end{tabular}\\
  \end{center}
  \item EE - A/AA - ExEN
  \begin{center}
    \begin{tabular}{|c|c|c|c|}
      \hline
      \textbf{Français} & \textbf{Verbe} & \textbf{Imparfait} & \textbf{Participe passé}\\
      \hline
      donner & geven & gaf/gaven & gegeven\\
      \hline
      manger & eten & at/aten & gegeten\\
      \hline
      oublier & vergeten & vergat/vergaten & vergeten\\
      \hline
      lire & lezen & las/lazen & gelezen\\
      \hline
    \end{tabular}\\
  \end{center}
  \item AA - OE - AxEN (ATTENTION AUX EXCEPTIONS)
  \begin{center}
    \begin{tabular}{|c|c|c|c|}
      \hline
      \textbf{Français} & \textbf{Verbe} & \textbf{Imparfait} & \textbf{Participe passé}\\
      \hline
      porter & dragen & droeg/droegen & gedragen\\
      \hline
      demander & vragen & vroeg/vroegen & GEVRAAGD\\
      \hline
      frapper & slaan & sloeg/sloegen & geslagen\\
      \hline
    \end{tabular}\\
  \end{center}
  \item AA - IE - AxEN
  \begin{center}
    \begin{tabular}{|c|c|c|c|}
      \hline
      \textbf{Français} & \textbf{Verbe} & \textbf{Imparfait} & \textbf{Participe passé}\\
      \hline
      laisser & laten & liet/lieten & gelaten\\
      \hline
      quitter & verlaten & verliet/verlieten & verlaten\\
      \hline
      dormir & slapen & sliep/sliepen & geslapen\\
      \hline
    \end{tabular}\\
  \end{center}
  \item A - AxTE - AxEN
  \begin{center}
    \begin{tabular}{|c|c|c|c|}
      \hline
      \textbf{Français} & \textbf{Verbe} & \textbf{Imparfait} & \textbf{Participe passé}\\
      \hline
      rire & lachen & lachte/lachten & gelachen\\
      \hline
      laver & wassen & waste/wasten & gewassen\\
      \hline
      cuire & bakken & bakte/bakten & gebakken\\
      \hline
    \end{tabular}\\
  \end{center}
  \item CHT
  \begin{center}
    \begin{tabular}{|c|c|c|c|}
      \hline
      \textbf{Français} & \textbf{Verbe} & \textbf{Imparfait} & \textbf{Participe passé}\\
      \hline
      chercher & zoeken & zocht & gezocht\\
      \hline
      se battre & vechten & vocht & gevocht\\
      \hline
      apporter & brengen & bracht & gebracht\\
      \hline
      penser & denken & dacht & gedacht\\
      \hline
      acheter & kopen & kocht/kochten & gekocht\\
      \hline
    \end{tabular}\\
  \end{center}
  \item Les vrais irréguliers
  \begin{center}
    \begin{tabular}{|c|c|c|c|}
      \hline
      \textbf{Français} & \textbf{Verbe} & \textbf{Imparfait} & \textbf{Participe passé}\\
      \hline
      faire & doen & deed/deden & gedaan\\
      \hline
      aller & gaan & ging/gingen & gegaan\\
      \hline
      venir & komen & kwam/kwamen & gekomen\\
      \hline
      avoir & hebben & had/hadden & gehad\\
      \hline
      être & zijn & was/waren & geweest\\
      \hline
      pouvoir (capacité) & kunnen & kon/konden & gekund\\
      \hline
      pouvoir (autorisation) & mogen &  mocht/mochten & gemogen\\
      \hline
      devoir & moeten & moest/moesten & gemoeten\\
      \hline
      courir & lopen & liep/liepen & gelopen\\
      \hline
      crier (appeller quelqu'un)& roepen & riep/riepen & geroepen\\
      \hline
      être debout & staan & stond/stonden & gestaan\\
      \hline
      tomber & vallen & viel/vielen & gevallen\\
      \hline
      savoir & weten & wist/wisten & geweten\\
      \hline
      devenir & worden &  werd/werden & geworden\\
      \hline
    \end{tabular}\\
  \end{center}
  \item Méthode pour essayer de se souvenir d'un verbe en Néerlandais : Passer par l'anglais
  \begin{itemize}
    \item find / vinden - found / vond - found / gevonden
    \item I -> IJ : ride / rijden
    \item OUGHT -> ACHT : brought / bracht
    \item etc...
  \end{itemize}
\end{itemize}

\section{Grammaire}
\subsection{Imparfait}
\begin{itemize}[label=\textbullet, font=\Large]
  \item Singulier
  \begin{itemize}[label=\ding{228}, font=\scriptsize]
    \item RADICAL + DE
  \end{itemize}
  \item Pluriel
  \begin{itemize}[label=\ding{228}, font=\scriptsize]
    \item RADICAL + DEN
  \end{itemize}
  \item Exemples
  \begin{itemize}[label=\ding{228}, font=\scriptsize]
    \item Horen (entendre)
    \begin{itemize}
      \item Ik hoorde
      \item We hoorden
    \end{itemize}
    \item Reizen (voyager)
    \begin{itemize}
      \item Ik reisde
      \item We reisden
    \end{itemize}
  \end{itemize}
  \item Exceptions
  \begin{itemize}[label=\ding{228}, font=\scriptsize]
    \item Si le radical (avant de le modifier, par exemple z -> s) se fini par "F K P S T C" (\textbf{F}ran\textbf{K}lin \textbf{P}rend \textbf{S}on \textbf{T}hé \textbf{C}haud), alors la terminaison sera TE (sing.) et TEN (pluriel)
    \item Exemple :
    \begin{itemize}
      \item Werken (travailler)
      \begin{itemize}
        \item Ik wer\textbf{k}te
        \item We wer\textbf{k}ten
      \end{itemize}
    \end{itemize}
  \end{itemize}
\end{itemize}
\subsection{Passé composé}
\begin{itemize}[label=\textbullet, font=\Large]
  \item Auxilaire au présent + participe passé
  \item Exemples :
  \begin{itemize}[label=\ding{228}, font=\scriptsize]
    \item Ik heb gegeten
  \end{itemize}
  \item Former un participe passé régulié :
  \begin{itemize}[label=\ding{228}, font=\scriptsize]
    \item \textbf{GE} devant la racine
    \item Sauf si commence par :
    \begin{itemize}
      \item be
      \item ge
      \item er 
      \item het 
      \item ont 
      \item ver 
    \end{itemize}
    \item Se termine par \textbf{d} ou \textbf{t} selon les mêmes règles que l'imparfait (ne pas ajouter si fini déjà par ses lettres)
    \item Si particule séparable, alors le \textbf{ge} se trouve entre la particule et le verbe :
    \begin{itemize}
      \item Opbellen --> op\textbf{ge}beld
    \end{itemize}
  \end{itemize}
\end{itemize}
\subsection{Futur}
\begin{itemize}[label=\textbullet, font=\Large]
  \item Auxilaire ZULLEN + infintif\\[0.2cm]
  \begin{tabular}{|c|c|}
      \hline
      \textbf{Sujet} & \textbf{Verbe}\\
      \hline
      Singulier & zal\\
      \hline
      Pluriel & zullen\\
      \hline
    \end{tabular}\\
  \item Exemple :
  \begin{itemize}[label=\ding{228}, font=\scriptsize]
    \item Hij zal werken
  \end{itemize}
\end{itemize}

\subsection{Conditionnel}
\begin{itemize}[label=\textbullet, font=\Large]
  \item Auxilaire ZULLEN à l'imparfait + infintif\\[0.2cm]
  \begin{tabular}{|c|c|}
      \hline
      \textbf{Sujet} & \textbf{Verbe}\\
      \hline
      Singulier & zou\\
      \hline
      Pluriel & zouden\\
      \hline
    \end{tabular}\\
  \item Exemple :
  \begin{itemize}[label=\ding{228}, font=\scriptsize]
    \item Als ik zou werken
  \end{itemize}
\end{itemize}

\subsection{Accord de l'adjectif}
\begin{itemize}[label=\textbullet, font=\Large]
  \item Adjectif + e
  \item Sauf si les 3 conditions sont remplies :
  \begin{itemize}[label=\ding{228}, font=\scriptsize]
    \item HET
    \item Singulier
    \item Indéterminé (een, geen, elk, welk, rien, etc...)
  \end{itemize}
\end{itemize}

\subsection{DE / HET}
\begin{itemize}[label=\textbullet, font=\Large]
  \item HET
  \begin{itemize}[label=\ding{228}, font=\scriptsize]
    \item Diminutifs
    \item Couleurs
    \item Matière (ou de métaux)
    \item Langues
    \item Pays
    \item Sports
    \item Noms qui viennent du latin (het museum a part de datum, de geranium, de petroleum)
    \item Territoire / Propriété
    \item Noms qui finissent par :
    \begin{itemize}
      \item tuig
      \item sel
      \item gram 
      \item ment 
      \item aal 
      \item isme 
      \item je
    \end{itemize}
    \item Noms qui commencent par :
    \begin{itemize}
      \item be
      \item ont
      \item ver 
      \item ge
    \end{itemize}
  \end{itemize}
  \item DE
  \begin{itemize}[label=\ding{228}, font=\scriptsize]
    \item Noms de personnes (de vrouw (sauf : kind, meisje))
    \item Noms de fleuves, montagnes, plantes, fruits, légumes (sauf pour het fruit)
    \item Noms de lettres et de chiffres (de vijf)
    \item Noms de sentiments
    \item Noms qui finissent par :
    \begin{itemize}
      \item ing
      \item heid
      \item is
      \item de
      \item te
      \item iek
      \item ica
      \item nis
      \item ie, tie, sie
      \item st
    \end{itemize}
  \end{itemize}
\end{itemize}

\subsection{Om...te / voor}
\begin{itemize}[label=\textbullet, font=\Large]
  \item Voor
  \begin{itemize}[label=\ding{228}, font=\scriptsize]
    \item Suivi d'un nom
  \end{itemize}
  \item Om...te
  \begin{itemize}[label=\ding{228}, font=\scriptsize]
    \item Suivi d'une action à l'infinitif
    \item Le om se situe au même endroit que le "pour" en français
    \item Le te se situe juste avant l'infinitif
  \end{itemize}
\end{itemize}

\subsection{Adjectif --> Nom}
\begin{itemize}[label=\textbullet, font=\Large]
  \item adjectif + heid
  \item Exemple :
  \begin{itemize}[label=\ding{228}, font=\scriptsize]
    \item Libre : vrij --> la liberté : de vrijheid
    \item Difficile : moeilijk --> la difficulté : de moeilijkheid
    \item Possible : mogelijk --> la possibilité : de mogelijkheid
    \item Facile : makkelijk --> La facilité : de makkelijkheid
  \end{itemize}
\end{itemize}

\subsection{Superlatifs}
\begin{itemize}[label=\textbullet, font=\Large]
  \item Beaucoup = veel
  \item Très = heel = zeer
\end{itemize}

\subsection{Verbes Pronominaux - ZICH}
\begin{tabular}{|c|c|}
  \hline
  \textbf{Sujet} & \textbf{Zich}\\
  \hline
  ik & me\\
  \hline
  je & je\\
  \hline
  hij & zich\\
  \hline
  we & ons\\
  \hline
  jullie & jullie/je\\
  \hline
  ze & zich\\
  \hline
\end{tabular}\\

\subsection{BE - verbe}
\begin{itemize}[label=\textbullet, font=\Large]
  \item Beluister
  \begin{itemize}[label=\ding{228}, font=\scriptsize]
    \item = luisteren naar
    \item Ik luister naar musiek = Ik beluister musiek
  \end{itemize}
  \item Bespreken
  \begin{itemize}[label=\ding{228}, font=\scriptsize]
    \item = discuter
  \end{itemize}
  \item Beschrijven
  \begin{itemize}[label=\ding{228}, font=\scriptsize]
    \item = décrire
  \end{itemize}
  \item Beantwoorden
  \begin{itemize}[label=\ding{228}, font=\scriptsize]
    \item = antwoorden op (répondre à)
    \item Ik antwoord op een vraag = Ik beantwoord een vraag
  \end{itemize}
\end{itemize}

\subsection{ON - verbe}
\begin{itemize}[label=\textbullet, font=\Large]
  \item Donne l'inverse du mot
  \item Exemple :
  \begin{itemize}[label=\ding{228}, font=\scriptsize]
    \item Mogelijk = Possible
    \item \textbf{On}mogelijk = Impossible
  \end{itemize}
\end{itemize}

\subsection{Pronoms démonstratif}
\begin{itemize}[label=\textbullet, font=\Large]
  \item Deze
  \begin{itemize}[label=\ding{228}, font=\scriptsize]
    \item pour des personnes ou choses proches
    \item Pour les DE-Woord
  \end{itemize} 
  \item Die
  \begin{itemize}[label=\ding{228}, font=\scriptsize]
    \item pour des personnes ou choses éloignées
    \item Pour les DE-Woord
  \end{itemize} 
  \item Dit
  \begin{itemize}[label=\ding{228}, font=\scriptsize]
    \item proche
    \item Pour les HET-Woord
  \end{itemize} 
  \item Dat
  \begin{itemize}[label=\ding{228}, font=\scriptsize]
    \item éloigné
    \item Pour les HET-Woord
  \end{itemize} 
\end{itemize}

\subsection{AAN - UIT}
\begin{itemize}[label=\textbullet, font=\Large]
  \item AAN
  \begin{itemize}[label=\ding{228}, font=\scriptsize]
    \item Faire l'action
    \item Pour les DE-Woord
    \begin{itemize}
      \item aankleden = habiller
    \end{itemize}
  \end{itemize} 
  \item UIT
  \begin{itemize}[label=\ding{228}, font=\scriptsize]
    \item Faire le contraire de l'action
    \item Exemple :
    \begin{itemize}
      \item uitkleden = déshabiller
    \end{itemize}
  \end{itemize} 
\end{itemize}

\subsection{Temps}
\begin{itemize}[label=\textbullet, font=\Large]
  \item hier = gisteren
  \item avant-hier = eergisteren
  \item demain = morgen
  \item après-demain = overmorgen
  \item matin = morgen/ochtend
  \item soir = avond
  \item demain matin = morgen ochtend
  \item Ce soir = vanavond
  \item Ce matin = vanmorgen
  \item Ce aprem = vannamiddag
  \item En début de phrase : 
  \begin{itemize}[label=\ding{228}, font=\scriptsize]
    \item s'Morgens
    \item s'Ochtends
    \item s'Avonds
  \end{itemize}
  \item Il y a
  \begin{itemize}
    \item Il y a 2 mois = 2 maanden geleden
    \item Il y a 2 ans = 2 jaren geleden 
  \end{itemize} 
  \item Heure
  \begin{itemize}
    \item 4h10 = tien over vier
    \item 3h45 = kwart voor vier
    \item 6h30 = half zeven
  \end{itemize}
\end{itemize}

\newpage
\section{Vocabulaire examen (pas exhaustif)}
\begin{center}
  \begin{tabular}{|l|l|}
    \hline
    \makecell[c]{\textbf{\large{Français}}} & \makecell[c]{\textbf{\large{Néerlandais}}}\\
    \hline
    Le village & het dorpje\\
    \hline
    Proche & nabij\\
    \hline
    Je fais de l'informatique & Ik doe aan informatica\\
    \hline
    le cheval & het paard\\
    \hline
    Bizarre & Vreemd\\
    \hline
    Le métier & Het beroep\\
    \hline
    Faire connaisance & kennismaken\\
    \hline
    La connaissance & de kennis\\
    \hline
    Se présenter & Voorstellen\\
    \hline
    Enchanté & Aangenaam\\
    \hline
    Se rencontrer & Ontmoeten\\
    \hline
    La rencontre & De ontmoeting\\
    \hline
    La présentation & De voorstelling\\
    \hline
    La connaissance de l'anglais & De kennis van het engels\\
    \hline
    Nous nous rencontrons & We ontmoeten elkaar\\
    \hline
    Nous faisons connaissance & We maken kennis van elkaar\\
    \hline
    Je me présente & Ik stel me voor\\
    \hline
    Tu te présentes & Je stelt je voor\\
    \hline
    J'aime la bière belge & Ik houd van Belgisch bier\\
    \hline
    Angleterre & Engeland\\
    \hline
    Allemagne & Duitsland \\
    \hline
    Enchanté & Aangenaam\\
    \hline
    Je marie Charlotte le mois prochain & Ik trouw Charlotte volgende maand \\
    \hline
    Je suis fidèle à Charlotte & Ik ben Charlotte trouw\\
    \hline
    Je le pensais & Ik dacht het\\
    \hline
    Je suis né en Belgique en décembre & Ik ben december in Belgïe geboren\\
    \hline
    Informatitien & Informaticus\\
    \hline
    Deux Informatitiens & Twee informatitie\\
    \hline
    Un flamand & Een vlaming\\
    \hline
    La liberté & De vrijheid\\
    \hline
    La difficulté & De moeilijkheid\\
    \hline
    La possibilité & De mogelijkheid\\
    \hline
    La facilité & De makelijkheid\\
    \hline
    La vérité & De waarheid\\
    \hline
    La responsabilité & De verantwoordelijkheid\\
    \hline
    Chaque jour & Elke dag\\
    \hline
    Je préfère courir & Ik loop liever\\
    \hline
    Le voyage & De reis\\
    \hline
    D'où viens-tu ? (Lieu) & Waar kom je vandaan ?\\
    \hline
    D'où viens-tu ? (Origine) & Waar kom je uit ?\\
    \hline
    Où vas-tu ? & Waar ga je naar te ?\\
    \hline
    Marié  / Epousé & Gehuwd / Getrouwd\\
    \hline
    Divorcé & Gescheiden\\
    \hline
    Le divorce & De scheiding\\
    \hline
    Saluer & Groeten \\
    \hline
    L'acceuil / La salutation & De groeting\\
    \hline
    Comment vas-tu ? & Hoe maak je uit ? \\
    \hline
    Je me présentais & Ik stelde me voor\\
    \hline
    Chaque jour & Elke dag\\
    \hline
    Je me présentais & Ik stelde me voor\\
    \hline
    Je me suis présenté & Ik heb me voorgesteld\\
    \hline
  \end{tabular}\\
\end{center}
\begin{center}
  \begin{tabular}{|l|l|}
    \hline
    A la campagne & Op het platteland\\
    \hline
    Flandre & Vlaanderen\\
    \hline
    Flamand (adj.) & Vlaamse\\
    \hline
    Wallonie & Wallonïe\\
    \hline
    Wallon (pers.) & Waal\\
    \hline
    Wallon (adj.) & Waals\\
    \hline
    Membres de la famille & Gezinsleden\\
    \hline
    Tôt & vroeg\\
    \hline
    A quelle heure te lèves-tu demain matin ? & Hoe laat sta je morgenochtend op ?\\
    \hline
    Célibataire & Vrijgezel\\
    \hline
    Je me sens bien & Ik voel me goed\\
    \hline
    Content & Blij\\
    \hline
    Je te présente Julien & Ik stel je Julien voor\\
    \hline
    A bientôt (dans la même journée) & Tot strachts\\
    \hline
    Je vous souhaite une bonne journée & Ik weens u een prettige dag\\
    \hline 
    Depuis 2001 & sinds 2001\\
    \hline
    Etage & verdieping\\
    \hline
    Employé & Bediende\\
    \hline
    Informations & inlichtingen\\
    \hline
    Moyens de transport & vervoermiddel\\
    \hline
    Sécurité & veiligheid\\
    \hline
    Poser une question & een vraag stellen\\
    \hline
    Je voudrais poser une question & Ik zou graag een vraag stellen\\
    \hline
    Navetteur & de pendelaar\\
    \hline
    Se déplacer (se rendre) & pendelen\\
    \hline
    Peur & bang\\
    \hline
    De qui as-tu peur ? & Voor wie ben je bang ?\\
    \hline
    Honnête & Eerlijk\\
    \hline
    Fier & Trots\\
    \hline
    La blague & De grap\\
    \hline
    Sentir & Voelen\\
    \hline
    La vie & het leven\\
    \hline
    L'amour & De liefde\\
    \hline
    La paresse & de luiheid\\
    \hline
    Le sentiment & het gevoel\\
    \hline
    Le calme & de rust\\
    \hline
    Réaliste & Realistisch\\
    \hline
    Ponctualité & De stiptheid\\
    \hline
    L'impatiente & Het ongeduld \\
    \hline
    S'ennuyer & Zich vervelen\\
    \hline
    Jamais & Nooit\\
    \hline
    Souvent & Vaak\\
    \hline
    Meestal & La plus part du temps\\
    \hline
    Toujours & Altijd\\
    \hline
    Parfois & Soms\\
    \hline
    Il a l'air grand & Hij ziet er groot uit\\
    \hline
    La différence & Het verschil \\
    \hline
    Je vais en ville & Ik ga weg naar de stad\\
    \hline
    Taille / Hauteur & De hoogte\\
    \hline
    La largeur & Breedte\\
    \hline
    Tu portes des lunettes & Je draagt een bril\\
    \hline
    Toujours & Altijd\\
    \hline
    Quand est votre anniversaire & Wanneer bent je jarig ?\\
    \hline
    Un jeune de 20 ans & A twintigjarig jongen\\
    \hline
    Je suis né le 9 décembre & Ik ben op negen december jarig\\
    \hline
  \end{tabular}\\
  \begin{tabular}{|l|l|}
    \hline
    C'est juste & Dat klopt\\
    \hline
    Mettre (vêtement) & aandoen\\
    \hline
    Enlever (vêtement) & uitdoen\\
    \hline
    S'habiller & zich aankleden \\
    \hline
    Se déshabiller & zich uitkleden\\
    \hline
    Les vêtements & de Kleren\\
    \hline
    La chemise & Het hemd\\
    \hline
    Il a mis sa veste & Hij doet zijn jas aan\\
    \hline
    Pantalon & De broek\\
    \hline
    Les chaussures & De schoenen\\
    \hline
    Moustache & De snor \\
    \hline
    Le chapeau & De hoed\\
    \hline
    Kaal & Chauve\\
    \hline
    Slank & Mince\\
    \hline
    Les jumeaux & de tweeling\\
    \hline
    A quoi ressemblent-ils ? & What zien ze eruit ?\\
    \hline
    Jupe & rok\\
    \hline
    Robe & De jurk\\
    \hline
    Conduire & Leiden\\
    \hline
    Le fondateur & De oprichter \\
    \hline
    Créer & Uitwerken\\
    \hline
    Le défi & De uitdaging\\
    \hline
    voix & stem\\
    \hline
    Il a allumé la lumière & Hij heeft het licht aangedaan\\
    \hline
    Il s'habille maintenant & Hij kleed zich nu aan\\
    \hline
    La TV est allumée/éteinte & De TV is aan/uit\\
    \hline
  \end{tabular}
\end{center}



























\end{document}